\input{preamble}


% OK, start here.
%
\begin{document}

\title{Introduction To Topological Manifolds}


\maketitle

\phantomsection
\label{section-phantom}

\tableofcontents

\section{Introduction on Topological Manifolds}
\label{section-introduction}

This is a test file which requires a reference to compile correctly. \cite{Maclane}

\section{Topological Spaces}
\label{section-topological-spaces}

\begin{exercise}
\label{tm-exercise-2-1}

Show that each of the following is a topological space.

\begin{enumerate}
\item Let $X$ denote the set $\{1, 2, 3\}$, and declare the open sets to be
$\{1\}$, $\{2, 3\}$, $\{1, 2, 3\}$, and the empty set.
\item Any set $X$ whatsoever, with $\tau = \{$ all subsets of $X\}$. This is
called the {\it discrete topology} on $X$, and $(X, \tau)$ is called the
{\it discrete space}.
\item Any set $X$, with $\tau = \{\emptyset, X\}$. This is called the {\it
trivial topology} on $X$.
\item Any metric space $(M, d)$, with $\tau$ equal to the collection of all
subsets of $M$ that are open in the metric space sense. This topology is called
the {\it metric topology} on $M$.
\end{enumerate}

\begin{proof}

For (1), both $X$ and the empty set are in $\tau$.
The union of any two elements is in $\tau$, the only case of interest being
$\{1\} \cup \{2, 3\} = X$. The intersection of any two is in $\tau$: any set
intersecting the empty set is the empty set and any set intersecting the entire
space is itself; the only case of interest is $\{1\} \cap \{2, 3\} = \emptyset$.

For (2), both $X$ and the empty set are subsets of $X$.
The union or intersection of any number of subsets of $X$ is again a subset of
$X$.

For (3), both $X$ and the empty set are in $\tau$.
The union of any two distinct elements is $X$; the intersection of any two
distinct elements is the empty set.

For (4), both $X$ and the empty set are open under the
metric $d$. If you take the union of any number of open sets, it is again open
as around any point we can take an open ball around it contained in the open
set contained in the union. The non empty intersection of any two open sets is
again open as we can take a point in the intersection and there is an open
ball containing the point for each open set, we just choose the smaller of the
two to make sure it is in both.

\end{proof}

\end{exercise}

\begin{multicols}{2}[\section{Other chapters}]
\noindent
Introduction
\begin{enumerate}
\item \hyperref[introduction-section-phantom]{Introduction}
\item \hyperref[conventions-section-phantom]{Conventions}
\item \hyperref[topological-manifolds-section-phantom]{Topological Manifolds}
\item \hyperref[smooth-manifolds-section-phantom]{Smooth Manifolds}
\item \hyperref[riemannian-manifolds-section-phantom]{Riemannian Manifolds}
\end{enumerate}
Miscellany
\begin{enumerate}
\setcounter{enumi}{82}
\item \hyperref[examples-section-phantom]{Examples}
\item \hyperref[exercises-section-phantom]{Exercises}
\item \hyperref[guide-section-phantom]{Guide to Literature}
\item \hyperref[desirables-section-phantom]{Desirables}
\item \hyperref[coding-section-phantom]{Coding Style}
\item \hyperref[obsolete-section-phantom]{Obsolete}
\item \hyperref[fdl-section-phantom]{GNU Free Documentation License}
\item \hyperref[index-section-phantom]{Auto Generated Index}
\end{enumerate}
\end{multicols}


\bibliography{my}
\bibliographystyle{amsalpha}

\end{document}
