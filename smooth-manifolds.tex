\input{preamble}


% OK, start here.
%
\begin{document}

\title{Introduction To Smooth Manifolds}


\maketitle

\phantomsection
\label{section-phantom}

\tableofcontents

This is a test file which requires a reference to compile correctly. \cite{Maclane}

\section{Smooth Manifolds}
\label{section-smooth-manifolds}

\begin{exercise}
\label{exercise-1-1}

Show that equivalent definitions of locally Euclidean spaces are obtained if
instead of requiring $U$ to be homeomorphic to an open subset of $\mathbb R^n$,
we require it to be homeomorphic to an open ball in $\mathbb R^n$, or to
$\mathbb R^n$ itself.

\end{exercise}

\begin{proof}

Let $p \in M$. If $U$ is homeomorphic to an open subset of $\mathbb R^n$, then
we can take an open ball around the image of $p$ via this homeomorphism and
then restrict the homeomorphism to the inverse image of this ball, yielding
another homeomorphism which is now homeomorphic to an open ball and in turn
homeomorphic to $\mathbb R^n$ in the natural way. The converse statement is
immediate by definition.

\end{proof}

\begin{exercise}
\label{exercise-1-2}

Show that $\mathbb R \mathbb P^n$ is Hausdorff and second countable and is
therefore a topological $n$-manifold.

\end{exercise}

\begin{proof}
Omitted.
\end{proof}


\begin{exercise}
\label{exercise-1-3}

Show that $\mathbb R \mathbb P^n$ is compact. [Hint: Show that the restriction
of $\pi$ to $\mathbb S^n$ is surjective.]

\end{exercise}

\begin{proof}
Omitted.
\end{proof}

\begin{lemma}
\label{lemma-1-10}

Let $M$ be a topological manifold.

\begin{enumerate}
\item Every smooth atlas $M$ is contained in a unique maximal smooth atlas.
\item Two snooth atlases $M$ determine the same maximal smooth atlas if an donly
if their union is a smooth atlas.
\end{enumerate}

\end{lemma}

\begin{proof}
Omitted.
\end{proof}

\begin{exercise}
\label{exercise-1-4}

Prove Lemma 1.10 (b)

\end{exercise}

\begin{proof}
Omitted.
\end{proof}

\begin{lemma}
\label{lemma-1-11}

Every smooth manifold has a countable basis of precompact smooth coordinate
balls.

\end{lemma}

\begin{proof}
Omitted.
\end{proof}

\begin{exercise}
\label{exercise-1-5}

Prove Lemma 1.11.

\end{exercise}

\begin{proof}
Omitted.
\end{proof}

\begin{proposition}
\label{proposition-1-25}

Let $M$ be a topological manifold with boundary.

\begin{enumerate}
\item $M$ is locally path connected.
\item $M$ has at most countably many components, each of whic his a connected
topological manifold with boundary.
\item The fundamental group of $M$ is countable.
\end{enumerate}

\end{proposition}

\begin{proof}
Omitted.
\end{proof}

\begin{exercise}
\label{exercise-1-6}

Prove Proposition 1.25.

\end{exercise}

\begin{proof}
Omitted.
\end{proof}


\section{Smooth Maps}
\label{section-smooth-maps}

\section{Tangent Vectors}
\label{section-tangent-vectors}

\section{Vector Fields}
\label{section-vector-fields}

\section{Vector Bundles}
\label{section-vector-bundles}

\section{The Cotangent Bundle}
\label{section-cotangent-bundle}

\section{Submersions, Immersions, and Embeddings}
\label{section-submersions-immersions-embeddings}

\begin{multicols}{2}[\section{Other chapters}]
\noindent
Introduction
\begin{enumerate}
\item \hyperref[introduction-section-phantom]{Introduction}
\item \hyperref[conventions-section-phantom]{Conventions}
\item \hyperref[topological-manifolds-section-phantom]{Topological Manifolds}
\item \hyperref[smooth-manifolds-section-phantom]{Smooth Manifolds}
\item \hyperref[riemannian-manifolds-section-phantom]{Riemannian Manifolds}
\end{enumerate}
Miscellany
\begin{enumerate}
\setcounter{enumi}{82}
\item \hyperref[examples-section-phantom]{Examples}
\item \hyperref[exercises-section-phantom]{Exercises}
\item \hyperref[guide-section-phantom]{Guide to Literature}
\item \hyperref[desirables-section-phantom]{Desirables}
\item \hyperref[coding-section-phantom]{Coding Style}
\item \hyperref[obsolete-section-phantom]{Obsolete}
\item \hyperref[fdl-section-phantom]{GNU Free Documentation License}
\item \hyperref[index-section-phantom]{Auto Generated Index}
\end{enumerate}
\end{multicols}


\bibliography{my}
\bibliographystyle{amsalpha}

\end{document}
